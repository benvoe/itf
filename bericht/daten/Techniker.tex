\chapter{Zusammenfassung für Techniker}
\label{sec:tec}




Die Beweismittel wurden personenweise auf die Ermittler aufgeteilt und mittels Autopsy, Sleuth Kit und Volatility gründlich untersucht. Während der Ermittlungen wurde bei allen Images nach Beweisen bezüglich aller Fälle gesucht.

\section{Jo}
\label{sec:jo}
Zunächst wurde der private USB-Stick untersucht, da dort die Besitzverhältnisse der Daten klar sind. Auf besagtem Beweisstück waren die gesuchten illegalen Daten in den Ordnern \textit{HighQuality} und \textit{Videos}. Bei der Untersuchung des Festplattenimages konnten die identischen Dateien in den \textit{carved Files} gefunden werden. Die Identität wurde mit Hashwerten überprüft, welche für die korrespondierenden Dateien erfolgreich validiert werden konnten. 

Darüber hinaus befanden sich auf dem Work USB-Stick im Ordner \textit{_TiggerTheCat} ebenfalls die gelöschten illegalen Dateien.

Eine Überprüfung der Zeitstempel ergab, dass die Dateien am 18.11.2009 auf den privaten USB-Stick kopiert wurden. Jo's Workstation wurde am 20.11.2009 ausgetauscht (s. \fref{sec:pat_emails_1}). Es kann daraus geschlussfolgert werden, dass die Dateien bereits vor dem Austausch in Jo's Besitz waren und die Dateien auf der verkauften Workstation  dementsprechend vermutlich von ihm stammen.

\section{Charlie}
\label{sec:charlie}
Die forensischen Untersuchungen von Charlie's Images ergaben einige Dateien, die beweisen, dass er sowohl in die Veruntreuung von Daten als auch in Erpressung verwickelt war. Die Daten wurden in Steganographie-Bilder versteckt und per E-Mail versandt. Die belastenden E-Mails und Bilder konnten sowohl auf der HDD im Ordner \textit{Emails} als auch auf dem USB-Stick gefunden werden. 

Bei der Veruntreuung der Daten handelte es sich um die Weitergabe von Ergebnissen bei einer Patentsuche für Nitroba im Projekt Time machines. Die veruntreuten Daten konnten unter \textit{mydocuments/nitrobe} gefunden werden. Zum Einbetten in das Bild \textit{Charlie_2009-12-03_1216_Sent_astronaut1.jpg} (\fref{sec:charlie_daten_4}) wurde das Steganographieprogramm Invisible Secrets gebraucht mit dem Passwort nitro. Der Empfänger der Daten war Jamie von project2400, welches die größten Konkurrenten von Nitroba sind. Charlie erhielt für diese Veruntreuung 50.000 Dollar. Die weiteren Beweise können in Anhang \ref{sec:charlie_daten} File 1 bis 6 eingesehen werden.

Die Erpressung bezog sich auf das Immortality Projekt von swexpert. Der Empfänger, Andy, der E-Mails wird auf 100.000 Dollar erpresst, da Charlie sonst Daten veröffentlicht, die das aktuelle Patent für Immortality ungültig lassen wird. Die Dateien werden als zip versendet und das zugehörige Passwort erneut als Steganographie-Bild (\fref{sec:charlie_daten_10}). Die Inhalte der zip-Datei konnten auch unverschlüsselt im Ordner \textit{Immortality} auf dem USB-Stick gefunden werden (im Anhang \ref{sec:charlie_daten} File 8).

\section{Terry}
\label{sec:terry}
Auf Terry'S USB-Stick konnten eine größere Ansammlung an Bildern gefunden werden, welche Screenshots vom Ausspionieren von Pat's Workstation zeigen. Des Weiteren konnte auf dem USB-Stick die Datei \textit{vnc-4_1_3-x86_win32.exe} gefunden werden. Eine Internetrecherche ergab, dass das Programm \textit{winvnc4} zum abhören und ausspionieren von Computern dient.

Auf seiner HDD konnte unter dem Pfad \textit{/Users/terry/Documents/Downloads/comcast_indictment.pdf} ein Gerichtsbeschluss über einen Wirtschaftsspionage-Fall gefunden werden, was darauf schließen lässt, dass er entweder auch in diesem Fall beteiligt war, oder sich über mögliche Konsequenzen informieren wollte.\newline

Darüber hinaus befanden sich Kontaktdaten von Terry's Handy auf seiner HDD. Diese zeigen, dass er die Kontaktdaten von Aaron Green, der Käuferin von Jo's alter Workstation abgespeichert hat.

\section{Pat}
\label{sec:pat}
Bei Pat konnte im RAM-Image der Prozess \textit{winvnc4.exe} mit der ProzessID \textit{756} am 11.12.2009 gefunden werden. Außerdem besitzt Terry ein Benutzerprofil auf Pat's Festplatte, allerdings nicht auf den Festplatten der anderen beiden Mitarbeiter. Diese beiden Tatsachen lassen darauf schließen, dass Terry Pat ausspioniert hat.

Auch die E-Mails im Anhang \ref{sec:pat_emails} FILE 14 bis 16 in denen Pat über einen langsamen Computer und ein fehlerhaftes Antiviren-Programm klagt und Terry ohne genauere Details sofort die Ursache kennt, deuten auf diesen Tatbestand hin.\newline

Zusätzlich zu den Kontaktdaten auf Terry's Festplatte existiert eine E-Mail an Pat (\fref{sec:pat_emeils_1}), in der Terry erklärt, dass er sich um die Workstation kümmern wird, legt es den Schluss nahe, dass Terry für den Verkauf der Workstation verantwortlich ist.

